يعتمد علم الحركة على دراسة حركة الأجسام ومختلف الأسباب الّتي تؤدّي إلا تغيّرها أو تعطّلها وتنقسم هذه المعرفة إلى عدّة أقسام منها ما يتعلّق ب الخصائص الرّياضيّة للمواقع والمسارات ومنها ما يتعلّق بالأسباب الّتي تؤدّي إلى تغيّر الحركة ومنها ما يتعلّق بماهيّة المكان والزّمان في علاقتهما بالحركة
\begin{section}{مفهوم الحركة}
الحركة مفهوم فطريّ يرتبط عند معظم النّاس بمفهوم مناقض له ألا وهو الثّبات أو السّكون فالشّيئ متحرّك ان لم يكن ثابتا وهو ثابت أوساكن إن لم يكن متحرّكا ولكنّ معنى الثّبات ضبابيّ وغير دقيق فهو عند الكثير من العوامّ مرتبط بالالتصاق بالأرض والتّشبّث بها فالأشجار بهذا المعنى ساكنة وكذلك الجبال والصّخور والبنايات بخلاف الحيوانات والسّيول والسّحب بل والشّمس والقمر وبعض النّجوم

ولكنّ المعاينة الموضوعيّة تحرج هذا الفهم وتلقي به عرضا لتلفّه الشّكوك ما إن تذكّر مخاطبك بأنّ اللأرض تدور حول الشّمس فإن كانت الأرض تدور حول الشّمس فهي متحرّكة وما تعلّق بمتحرّك فلا يكون ثابتا

ثمّ إنّ القابع في قطار أو في غرفة على ظهر سفينة قد يخيّل إليه أنّ الأشياء ثابتة لا تتحرّك بيدأنّها قد تكون متحرّكة أو ساكنة حسب حالة وسيلة النّقل الّتي تحويها
\begin{subsection}{نسبيّة الحركة}
ظهر من الأمثلة المذكورة أعلاه أنّ مفهومي السّكون والحركة ليسا بالبديهيّة الّتي قد تظنّ من الوهلة الأولى وذلك لارتباطها بالملاحِظ وحالته فالمتحرّك بالنّسبة لأحد ما قد يكون ساكنا بالنّسبة للآخر والعكس صحيح أيظا ولكن البعض قد يعترض على هذا القول بدليل أنّ المنطق السّليم ل يجيز الشّيئ وضدّه في آن واحد فكيف يعقل أنّ رجلا واقفا أراه ساكنا وتراه أنت متحرّكا يقطع المسافات وينتقل من مكان لآخر؟

الإجابة قد تكون أسهل ممّا نتوقّع ويكفي أن نتخيّل أنَنا في عربة قطار بها شبابيك مغلقة وقد غفونا قليلا ثمّ استيقظنا فنظرنا إلى ساعتنا ثمّ إلى تذكرة السّفر فتنبّأنا أنّ القطار قد النطلق منذ نصف ساعة وأنّه لابدّ أن يكون قد بلغ أقصى سرعته ولكنّنا لا نشعر بحركته فالمقصورة من النّوع الفاخر جدرانها عازلة للصّوت ونوافذها محكمة الإغلاق لا يدخل منها بصيص من الضّوء والطّريق منبسطة فنشكّ أنّ القطار قد تأخّر عن موعد انطلاقه ولكنّ الفضول يدفعنا للتّأكّد من أنّه لم ينطلق بعد فنحاول فتح النّافذة ولكنّ العربة الفاخرة مجهّزة بنوافذ تفتح بآلة تحكّم عن بعد ونحن لا نجدها فنتشاور ونفكّر في طريقة لنكتشف حالة القطار دون أن نخرج من العربة المكيّفة لشدّة الحرّ نبحث في جيوبنا فنجد قطعا نقديّة فنفكّر أنّنا إن ألقينا بهاإلى الأعل فسوف نراها في حال كان القطار ساكنا ترتفع في حركة عموديّة ثمّ تقع في أيدينا أمّا إذا كان القطار متحرّكا فإنّ القطعة الملقاة في الهواء سوف ترتفع في الهواء عموديّابينما يتقدّم القطار أفقيّا فتسقط القطعة بعيدا عن اليد الّتي ألقتها واللّتي تتحرّك بسرعة القطار لاتّصالها بأرضيّة القطار
\begin{center}
\begin{minipage}{\textwidth*3/4}
\begin{small}
لنفترض أنّ القطعة النّقديّة في حالة سكون بالنّسبة للأرض فعند إلقائها عموديّا فإنّها سترتفع في السّماء وتنخفض سرعتها بدريجيّا حتّى تتوقّف تماما ثمّ تبدأ في الهبوظ حتّى تقع على سطح الأرض في نفس النّقطة الّتي انطلقت منها

أثناء ذلك يكون القطار قد تقدّم المسافة
\begin{amath}
\abox{م}_{\abox{\tiny القطار}} = \vec{\abox{س}}_{\abox{\tiny القطار}} \times \abox{ز}
\end{amath}
\end{small}
\end{minipage}
\end{center}

نقوم بإلقاء القطعة فترتفع ثمّ تقع في اليد الّتي ألقتها فنستنتج أنّ القطار لم ينطلق بعد وبينما نحن في أخذ وردّ حول سبب تأخّر القطار إذا بمراقب التّذاكر يمرّ فنسأله عن السّبب فيتعجّب ويخبرنا أنّ القطار قد انطلق في موعده تماما وهو يسير بسرعة 400 كم في السّاعة ثمّ يقوم برفع السّتار لنرى فجأة وكأنّ الأشجار تتطاير أمامنا وكأنّ الأرض قد أصابها الجنون فأخذت تتحرّك أمامنا بسرعة رهيبة ثمّ ننتبه إلى أننّا نحن من بتحرّك مع القطار
\end{subsection}
\end{section}
